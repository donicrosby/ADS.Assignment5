\documentclass{scrreprt}

% Document information
%\title{Advance Digital Systems: Assignment 1}
%\subtitle{Group 9}
%\author{Doni Crosby \and Grant Moyer}
%\date{2016--12--13}

% Use Palatino Font
\usepackage{tgpagella,eulervm}

% Use 1 in margins
\usepackage[letterpaper, margin=1in]{geometry}

% fixme notes
\usepackage{fixme}
\fxsetup{
    status=draft,
    layout=margin,
}

% Unit formating
\usepackage[binary-units, detect-all=true]{siunitx}

% mars symbol
\usepackage[nointegrals]{wasysym}

% diagonal fraction
\usepackage{xfrac}

% Rotations for figure
\usepackage{rotating}

% Refrence Links
\usepackage{hyperref}
\usepackage{bookmark}

% No section numbers
\makeatletter
\renewcommand\@seccntformat[1]{}
\makeatother

% Refrences
\newcommand{\FigLabel}[1]{\label{fig:#1}}
\newcommand{\Figure}[1]{Figure~\ref{fig:#1}}
\newcommand{\TabLabel}[1]{\label{tab:#1}}
\newcommand{\Table}[1]{Table~\ref{tab:#1}}

% Figure arrays
\usepackage{subcaption}

% No paragraph indents
\usepackage{parskip}

% Better math
\usepackage{amsmath}

% Boolean operators
\DeclareMathOperator{\NOT}{\neg}
\DeclareMathOperator{\XOR}{\oplus}
\DeclareMathOperator{\AND}{\wedge}
\DeclareMathOperator{\OR}{\vee}
\DeclareMathOperator{\NAND}{\bar{\wedge}}
\DeclareMathOperator{\NOR}{\bar{\vee}}

% Source for figures (if needed)
\newcommand{\source}[1]{\caption*{Source: {#1}} }

% Figures, equations, and tables don't use chapter numbers
\usepackage{chngcntr}
\counterwithout{figure}{chapter}
\counterwithout{table}{chapter}
\counterwithout{equation}{chapter}

% Component counts
\newcommand{\component}[2]{(#1) #2}

% Insert images
\usepackage{graphicx}

% Weird FTQ number
\newlength\Wwidth
\setlength\Wwidth{1.1pt}
\newlength\Wheight
\settoheight\Wheight{H}
\newcommand{\WeirdNumber}{%
    \rule{\Wwidth}{\Wheight}\hspace{\Wwidth}%
    \rule{\Wwidth}{\Wheight}\hspace{\Wwidth}%
    \rule{\Wwidth}{\Wheight}\hspace{\Wwidth}%
    \rule{\Wwidth}{\Wheight}%
}



\usepackage{fontspec}
\setsansfont[
BoldFont=OpenSans-Bold.ttf,
ItalicFont=OpenSans-Italic.ttf,
BoldItalicFont=OpenSans-BoldItalic.ttf
]{OpenSans-Regular.ttf}

\usepackage{xltxtra}
\newfontface\emojiFont {seguisym.ttf} % Segoe UI Sybol should have even better support

\catcode `\ᚊ=\active
\protected\def ᚊ{\leavevmode{\emojiFont ᚊ}}

\catcode `\🕚=\active
\protected\def 🕚{\leavevmode{\emojiFont 🕚}}



% Allows code to be easily formatted and compiled
\usepackage{listings}
\usepackage{color}
\lstset{frame=tb,
  language=VHDL,
  aboveskip=3mm,
  belowskip=3mm,
  showstringspaces=false,
  columns=flexible,
  basicstyle={\small\ttfamily},
  numbers=none,
  numberstyle=\tiny\color{gray},
  keywordstyle=\color{blue},
  commentstyle=\color{dkgreen},
  stringstyle=\color{mauve},
  breaklines=true,
  breakatwhitespace=true,
  tabsize=3
}
% Allows for importation of unicode characters

%\DeclareUnicodeCharacter{1F55A}{\grantFTQ}

\begin{document}
%\maketitle

% Title Page
\begin{titlepage}
\newcommand{\HRule}{\rule{\linewidth}{0.5mm}} % Defines a new command for the horizontal lines, change thickness here

\center % Center everything on the page
 
%----------------------------------------------------------------------------------------
%	HEADING SECTIONS
%----------------------------------------------------------------------------------------

\textsc{\LARGE University of Massachusetts at Lowell}\\[1.5cm] % Name of your university/college
\textsc{\Large Advanced Digital Systems}\\[0.5cm] % Major heading such as course name
\textsc{\large Group 9}\\[0.5cm] % Minor heading such as course title

%----------------------------------------------------------------------------------------
%	TITLE SECTION
%----------------------------------------------------------------------------------------

\HRule \\[0.4cm]
{ \huge \bfseries Assignment 5:\\ Introduction to FGPAs \& Tutorial}\\[0.4cm] % Title of your document
\HRule \\[1.5cm]
 
%----------------------------------------------------------------------------------------
%	AUTHOR SECTION
%----------------------------------------------------------------------------------------

\Large \emph{Authors:}\\
Doni \textsc{Crosby}\\
Grant \textsc{Moyer}\\[3cm]

%----------------------------------------------------------------------------------------
%	DATE SECTION
%----------------------------------------------------------------------------------------

{\large December 13, 2016}\\[3cm] % Date, change the \today to a set date if you want to be precise

%----------------------------------------------------------------------------------------
%	LOGO SECTION
%----------------------------------------------------------------------------------------

%\includegraphics{Logo}\\[1cm] % Include a department/university logo - this will require the graphicx package
 
%----------------------------------------------------------------------------------------

\vfill % Fill the rest of the page with whitespace

\end{titlepage}

\section{Objective}
   The objective of this project was to follow the tutorials for the Altera DE2 development board to understand how to program the FPGA using VHDL.

\section{Equipment}
    % \subsection{Active components}
    %     \begin{itemize}
    %         \item \component{1}{SN7489 16~address by \SI{4}{\bit} RAM}
    %         \item \component{1}{74LS157 Quad \(2 \times 1\) multiplexer}
    %         \item \component{1}{74LS181 ALU}
    %         \item \component{1}{74LS195 \SI{4}{\bit} shift register}
    %         \item \component{1}{74LS569 \SI{4}{\bit} counter}
    %         \item \component{2}{74LS157 BCD to seven segment decoder}
    %         \item \component{1}{LN526RA Two digit seven segment display}
    %     \end{itemize}
    
    % \subsection{Passive Components}
    %     \begin{itemize} 
    %         \item Various pull up resistors
    %         \item \component{2}{17DIP8SS 16 Pin dip switch}
    %         \item \component{1}{Toggle switch}
    %     \end{itemize}
    
    \subsection{Lab Equipment}
        \begin{itemize}
            \item Altera DE2 FPGA Devoloplment Board
        \end{itemize}

\section{Experimental Approach}
    The assignment consisted of 6 parts, I through VI. Each part built upon the previous part, either by directly using components from the previous part or by extending concepts covered in the previous part.
    
    Part~I introduced connecting ports and signals together by requiring each switch, SW0 through SW17, be connected to a built in LED. in VHDL, an connection is represented as a signal assignment: \lstinline{signal1 <= signal2}.
    
    Part~II built off of Part~I by connecting switches to a multiplexer. The multiplexer was made with behavioral VHDL:
    
    \begin{lstlisting}
    %put the code here.
    \end{lstlisting}
    
    

\section{Results}
    All parts worked as indended.
    
    \begin{table}[ht]
        \centering
        \begin{tabular}{|l|c|c|l|} \hline
            \bf Instruction & \multicolumn{1}{|l|}{\bf ACC Display} & \multicolumn{1}{|l|}{\bf PC Display} & \bf Notes  \\ \hline
            LDI 3   & 3  & 1 & \\ \hline
            STORE 2 & 3  & 2 & \\ \hline
            LDI 4   & 4  & 3 & \\ \hline
            LOAD 2  & 3  & 4 & \\ \hline
            LDI 9   & 9  & 6 & incorrect instruction entered, not shown\\ \hline
            SUB 2   & 7  & 7 & \\ \hline
            JIFP    & 7  & 8 & \\ \hline
            ADDI 4  & 11 & 9 & ADDI instruction was originally ANDI\\ \hline
            JMPZ    & 11 & 0 & \\ \hline
            LDI 0   & 0  & 1 & \\ \hline
            JIFZ    & 0  & 0 & \\ \hline
        \end{tabular}
        \caption{Sample Instruction Sequence}
        \TabLabel{sample}
    \end{table}
    
\section{Conclusions}
    If possible, a processor implementation can be greatly simplified by carefully choosing the instruction encoding; the more similar the instructions are to the control words, the simpler the implementation. For example, some complexity in design arose because a "jump" bit could not be included in the instruction. Whether or not an instruction was a jump needed to be determined through combinational logic.

\section{Fault Tolerance Questions}
    \subsection{Doni}
        \subsubsection{ᚊ. Perform the following instruction code and explain the result.}
            {\centering%
            \begin{tabular}{|c|c|c|c|c|c|c|} \hline
                \multicolumn{2}{|c|}{\bf Odd Byte} & \multicolumn{5}{c|}{\bf Even Byte}  \\ \hline
                DB   & RAM  & MUX & \(\overline{\text{WE}}\) & \Bit[n]{C} & M & S    \\ \hline
                1001 & 0000 & 1   & 0                        & 1          & 1 & 0011 \\ \hline
            \end{tabular}\par
            }
            
            After a reset of the processor and inputting the control word from above. The if M is high it causes the ALU to be in the logic mode rather than arithmetic. That means that the \Bit[n]{C} is irrelevant as it becomes a don't care. The function selected is the F=1, there is no data read in from the MUX. Because of the selected function it will output the same output as in the previous function or F=1.

    \subsection{Grant}
        \subsubsection{🕚. Perform the following instruction code and explain the result.}
            {\centering%
            \begin{tabular}{|c|c|c|c|c|c|c|} \hline
                \multicolumn{2}{|c|}{\bf Odd Byte} & \multicolumn{5}{c|}{\bf Even Byte}  \\ \hline
                DB   & RAM  & MUX & \(\overline{\text{WE}}\) & \Bit[n]{C} & M & S    \\ \hline
                0100 & 1101 & 1   & 1                        & 0          & 1 & 0000 \\ \hline
            \end{tabular}\par
            }
            
            After resetting the processor, performing the operation results in an accumulator containing F and a process counter containing 1. Repeating the instruction results in the accumulator alternating between 0 and F, and the process counter incrementing. This is because S, the operation code, along with M, the mode bit, represents the ALU operation \(\NOT{A}\). Since the \(A\) operand is the accumulator, the instruction inverts the accumulator.

            
\end{document}
